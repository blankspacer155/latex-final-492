\chapter{\ifenglish Introduction\else บทนำ\fi}

\section{\ifenglish Project rationale\else ที่มาของโครงงาน\fi}

ในปัจจุบันปัญหาด้านสุขภาพของประชากรมีแนวโน้มจะสูงขึ้นเรื่อย ๆ ประกอบกับการเข้าสู่สังคมสูงวัยของประชากร ปัญหาสุขภาพจึงเป็นปัญหาที่สำคัญ ซึ่งส่งผลกระทบต่อชีวิตของประชากรโดยส่วนมาก มะเร็งช่องปากเป็นมะเร็งชนิดหนึ่งที่พบมากในกลุ่มประชากรที่มีอายุตั้งแต่ 40 ปี ขึ้นไป ที่มีประวัติด้านการสูบบุหรี่ และ/หรือดื่มแอลกอฮอล์ และเคี้ยวหมาก ซึ่งการตรวจสอบรอยโรคในระยะแรกอาจทำได้ยากโดยทั่วไปและหากปล่อยเป็นระยะเวลานานเกินไปอาจทำให้รอยโรคลุกลามเป็นมะเร็งได้ในที่สุด 

คณะผู้จัดทำมีความสนใจในเรื่องนี้ จึงได้จัดทำดิจิทัลแพลตฟอร์มเพื่อรองรับระบบปัญญาประดิษฐ์ (AI) เพื่อใช้ในการตรวจสอบและคัดกรองรอยโรคก่อนมะเร็งช่องปากและมะเร็งช่องปาก ที่ใช้ร่วมกับการประเมินจากทันตแพทย์ผู้เชี่ยวชาญ คณะผู้จัดทำจึงได้นำเสนอการพัฒนาเว็บแอปพลิเคชัน ซึ่งเป็นแพลตฟอร์มดิจิทัลสำหรับรองรับระบบปัญญาประดิษฐ์ (AI) เพื่อตรวจคัดกรองและเฝ้าระวังการเกิดรอยโรคก่อนมะเร็งและมะเร็งช่องปาก (Digital Platform for Detecting and Analyzing Oral Potentially Malignant Disorders and Oral Cancer) โดยกลุ่มผู้ใช้งานของดิจิทัลแพลตฟอร์มนี้จะเป็นทันตแพทย์ทั่วประเทศและประชากรทั่วไปที่มีความสนใจในการนำดิจิทัลแพลตฟอร์มนี้ไปใช้ 

คณะผู้จัดทำหวังว่า ดิจิทัลแพลตฟอร์มนี้จะส่งผลให้ทันตแพทย์ทั่วประเทศสามารถตรวจหามะเร็งช่องปากได้อย่างรวดเร็ว และเป็นเครื่องมือหนึ่งที่จะช่วยแก้ไขปัญหาด้านสุขภาพของประชากรโดยเฉพาะมะเร็งช่องปากได้อย่างมีประสิทธิภาพ
\section{\ifenglish Objectives\else วัตถุประสงค์ของโครงงาน\fi}
\begin{enumerate}
    \item พัฒนาเว็บแอปพลิเคชันเพื่อรองรับระบบปัญญาประดิษฐ์(AI)
    \item พัฒนาเว็บแอปพลิเคชันเพื่อตรวจคัดกรองและเฝ้าระวังการเกิดรอยโรคก่อนมะเร็งและมะเร็งช่องปาก
\end{enumerate}

\section{\ifenglish Project scope\else ขอบเขตของโครงงาน\fi}

\subsection{\ifenglish Hardware scope\else ขอบเขตด้านฮาร์ดแวร์\fi}
โครงการนี้ต้องการฮาร์ดแวร์ต่อไปนี้ จึงจะสามารถใช้งานได้อย่างมีประสิทธิภาพ

• คอมพิวเตอร์ส่วนบุคคลหรือโทรศัพท์มือถือที่สามารถใช้งานเว็บเบราว์เซอร์ได
\subsection{\ifenglish Software scope\else ขอบเขตด้านซอฟต์แวร์\fi}
โครงการนี้ต้องการซอฟต์แวร์ต่อไปนี้ จึงจะสามารถใช้งานได้อย่างมีประสิทธิภาพ

• สามารถใช้งานเว็บไซต์บนระบบปฏิบัติการทั่วไปได้ เช่น Windows, macOS, Linux, Android, iOS และอื่น ๆ

\section{\ifenglish Expected outcomes\else ประโยชน์ที่ได้รับ\fi}
ผู้ใช้งาน

• สามารถใช้งานเว็บแอปพลิเคชันเพื่อตรวจคัดกรองและเฝ้าระวังการเกิดรอยโรคก่อนมะเร็งและมะเร็งช่อง
ปากได้

• สามารถเข้าถึงการรักษาทางการแพทย์ได้อย่างรวดเร็ว หลังจากที่ผู้ใช้งานได้รับการตรวจคัดกรองและ
เฝ้าระวังการเกิดรอยโรคก่อนมะเร็งและมะเร็งช่องปากโดยเว็บแอปพลิเคชัน

\noindent ผู้พัฒนา

• ได้รับความรู้และความเข้าใจในการพัฒนาเว็บแอปพลิเคชันเพื่อรองรับระบบปัญญาประดิษฐ์(AI)

• ได้ฝึกทักษะในการพัฒนาเว็บแอปพลิเคชันเพื่อรองรับระบบปัญญาประดิษฐ์(AI)

• ได้ฝึกทักษะในการทํางานเป็นทีมและทักษะในการวิเคราะห์และแก้ไขปัญหาที่อาจเกิดขึ้นในการพัฒนา

\section{\ifenglish Technology and tools\else เทคโนโลยีและเครื่องมือที่ใช้\fi}

\subsection{\ifenglish Hardware technology\else เทคโนโลยีด้านฮาร์ดแวร์\fi}

\subsection{\ifenglish Software technology\else เทคโนโลยีด้านซอฟต์แวร์\fi}

• ภาษาโปรแกรมมิ่ง: JavaScript, Python, HTML, CSS

• ฐานข้อมูล: MySQL

• เครื่องมือและเทคโนโลยี: NextJS, Tailwind CSS, Git, GitHub, Minio
\section{\ifenglish Project plan\else แผนการดำเนินงาน\fi}

\begin{plan}{11}{2023}{2}{2024}
    \planitem{11}{2023}{1}{2024}{ศึกษาค้นคว้าเกี่ยวกับเทคโนโลยีที่เกี่ยวข้อง}
    \planitem{12}{2023}{12}{2023}{ออกแบบฟีเจอร์ที่จะเพิ่มในเว็บแอปพลิเคชัน}
    \planitem{1}{2024}{2}{2024}{พัฒนาฟีเจอร์ตามที่ได้ออกแบบไว้}
    \planitem{2}{2023}{2}{2024}{ทดสอบและปรับปรุง}
\end{plan}

\section{\ifenglish Roles and responsibilities\else บทบาทและความรับผิดชอบ\fi}

มีหน้าที่และความรับผิดชอบ ดังนี้

นายชาญชัย ไชยสลี รหัสนักศึกษา 630610726 รับผิดชอบในการศึกษาค้นคว้าเทคโนโลยีที่เกี่ยวข้อง,
ออกแบบโครงสร้างของเว็บแอปพลิเคชันและพัฒนาเว็บแอปพลิเคชัน

นายเทวฤทธิ์ สมฤทธิ์ รหัสนักศึกษา 630610731 รับผิดชอบในการศึกษาค้นคว้าเทคโนโลยีที่เกี่ยวข้อง,
ออกแบบโครงสร้างของเว็บแอปพลิเคชันและพัฒนาเว็บแอปพลิเคชัน


\section{\ifenglish%
Impacts of this project on society, health, safety, legal, and cultural issues
\else%
ผลกระทบด้านสังคม สุขภาพ ความปลอดภัย กฎหมาย และวัฒนธรรม
\fi}

โครงงานนี้จัดทำขึ้นเพื่อให้ประชาชนทั่วไปได้มีโอกาสเข้าถึงการรักษาได้มากยิ่งขึ้น เพิ่มโอกาสการรักาาสำเร็จให้สูงขั้นได้
ถ้าตรวจพบโรคได้เร็วพอ แต่ทั้งนี้ก็ต้องอาศัยความร่วมมือจากหน่วยงานทางการแพทย์ที่นำไปใช้งานจริง
