\chapter{\ifproject%
\ifenglish Experimentation and Results\else การทดลองและผลลัพธ์\fi
\else%
\ifenglish System Evaluation\else การประเมินระบบ\fi
\fi}
% \chapter{\ifenglish System Evaluation\else การประเมินระบบ\fi}

ในบทนี้จะทดสอบเกี่ยวกับการทำงานในฟังก์ชันหลักๆ

\section{การทดลองเกี่ยวกับการทํางานของระบบ}

การประเมินระบบจะประเมินโดยทดสอบกับกลุ่มผู้ใช้งานทั้ง 4 กลุ่ม ได้แก่ ผู้ใช้ทั่วไป, ทันตแพทย์, ทันตบุคลากร และอาสาสมัครสาธารณสุขประจําหมู่บ้าน (อสม.) โดยในการทดสอบระบบจะมีการประเมินผลการ
ทดลองโดยใช้เกณฑ์ต่าง ๆ ดังนี้

\subsection{ผู้ใช้ทั่วไป}

ผู้ใช้ทั่วไปมักมีความต้องการใช้งานระบบที่เรียบง่าย ใช้งานง่าย ไม่ซับซ้อน ดังนั้น ในการทดสอบกับผู้ใช้ทั่วไป
ควรเน้นการประเมินปั จจัยต่างๆ เช่น

• ความน่าใช้งาน: Ease of use

• ความพึงพอใจของผู้ใช้งาน: User satisfaction


• ประโยชน์: Benefits

ตัวอย่างวิธีการทดสอบกับผู้ใช้ทั่วไป ได้แก่

• ให้ผู้ใช้ทดสอบระบบและรวบรวมข้อมูลเกี่ยวกับประสบการณ์การใช้งาน เช่น ระยะเวลาในการดําเนิน
การแต่ละขั้นตอน ความสะดวกในการใช้งาน เป็นต้น

• ให้ผู้ใช้ตอบแบบสอบถามเกี่ยวกับความพึงพอใจต่อระบบ เช่น ความง่ายในการใช้งาน ความน่าสนใจ
ของเนื้อหา เป็นต้น

• ให้ผู้ใช้ประเมินประโยชน์ที่ได้รับจากระบบ เช่น ช่วยให้ประหยัดเวลา ช่วยให้เข้าใจข้อมูลต่างๆ ได้ง่าย
เป็นต้น

\subsection{ทันตแพทย์}

ทันตแพทย์มีความต้องการใช้งานระบบที่มีประสิทธิภาพ ถูกต้องแม่นยําและสามารถช่วยในตรวจคัดกรองมะเร็งช่องปากได้อย่างมีประสิทธิภาพ ดังนั้น ในการทดสอบกับทันตแพทย์ ควรเน้นการประเมินปัจจัยต่างๆ เช่น

• ความน่าใช้งาน: Ease of use

• ความพึงพอใจของผู้ใช้งาน: User satisfaction

• ประโยชน์: Benefits

• การช่วยในการตรวจคัดกรองมะเร็งช่องปาก: Screening

ตัวอย่างวิธีการทดสอบกับทันตแพทย์ ได้แก่

• ให้ทันตแพทย์ทดสอบระบบภายใต้สถานการณ์จริง เช่น ถ่ายภาพช่องปากของผู้ป่ วย และให้ระบบตรวจ
คัดกรอง และให้ทันตแพทย์ประเมินความถูกต้องแม่นยําของระบบ เป็นต้น

• ให้ทันตแพทย์ประเมินประโยชน์ที่ได้รับจากระบบ เช่น ช่วยให้ตรวจคัดกรองมะเร็งช่องปากได้อย่างมี
ประสิทธิภาพหรือไม่ เป็นต้น

\subsection{ทันตบุคลากร}

ทันตบุคลากรมีความต้องการใช้งานระบบที่อํานวยความสะดวกในการทํางาน เช่น การดูประวัติการตรวจคัด
กรองมะเร็งช่องปาก การบันทึกข้อมูล การสรุปผลการตรวจคัดกรองมะเร็งช่องปาก ดังนั้น ในการทดสอบกับ
ทันตบุคลากร ควรเน้นการประเมินปัจจัยต่างๆ เช่น

• ความสะดวกในการใช้งาน: Ease of use

• ประโยชน์: Benefits

ตัวอย่างวิธีการทดสอบกับทันตบุคลากร ได้แก่

• ให้ทันตบุคลากรทดสอบระบบและรวบรวมข้อมูลเกี่ยวกับประสบการณ์การใช้งาน เช่น ระยะเวลาใน
การดําเนินการแต่ละขั้นตอน ความสะดวกในการใช้งาน เป็นต้น

• ให้ทันตบุคลากรประเมินประโยชน์ที่ได้รับจากระบบ เช่น ระบบช่วยให้ทํางานได้อย่างมีประสิทธิภาพ
หรือไม่ เป็นต้น

\subsection{อาสาสมัครสาธารณสุขประจําหมู่บ้าน (อสม.)}
อสม. มีความต้องการใช้งานระบบที่เข้าใจง่าย ใช้งานสะดวก และสามารถช่วยให้ให้บริการประชาชนได้อย่าง
มีประสิทธิภาพ ดังนั้น ในการทดสอบกับอสม. ควรเน้นการประเมินปัจจัยต่าง ๆ เช่น

• ความน่าใช้งาน: Ease of use

• ความพึงพอใจของผู้ใช้งาน: User satisfaction

• ประโยชน์: Benefits

ตัวอย่างวิธีการทดสอบกับอสม. ได้แก่


• ให้อสม.ทดสอบระบบและรวบรวมข้อมูลเกี่ยวกับประสบการณ์การใช้งาน เช่น ระยะเวลาในการดําเนิน
การแต่ละขั้นตอน ความสะดวกในการใช้งาน เป็นต้น

• ให้อสม.ตอบแบบสอบถามเกี่ยวกับความพึงพอใจต่อระบบ เช่น ความง่ายในการใช้งาน ความน่าสนใจ
ของเนื้อหา เป็นต้น

• ให้อสม.ประเมินประโยชน์ที่ได้รับจากระบบ เช่น ระบบช่วยให้บริการประชาชนได้อย่างมีประสิทธิภาพหรือไม่ เป็นต้น

ทั้งนี้ ในการทดสอบระบบกับผู้ใช้ทั่วไป, ทันตแพทย์, ทันตบุคลากรและอสม จะพิจารณาจากปัจจัยต่าง ๆ
เช่น วัตถุประสงค์ของการประเมิน ขอบเขตของการประเมิน ความพร้อมของระบบ เป็นต้น เพื่อให้ได้ผลการ
ประเมินที่มีประสิทธิภาพ