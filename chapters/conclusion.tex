\chapter{\ifenglish Conclusions and Discussions\else บทสรุปและข้อเสนอแนะ\fi}

\section{\ifenglish Conclusions\else สรุปผล\fi}

ในการทำโครงงานนี้ สามารถพัฒนาเว็บแอปพลิเคชัน ที่สามารถทำงานร่วมกับระบบAIได้จริง
โดยมีความแม่นยำในการทำนายรอยโรคในระดับที่พึงพอใจ แต่สามารถพัฒนาให้ดียิ่งขึ้นกว่านี้ได้อีกทั้งในส่วนของ
AIและส่วนของระบบเว็บแอปพลิเคชัน ที่สามารถทำให้มีความพึงพอใจในการใช้งานมากขึ้นได้ เพิ่มให้มีความสวยงามและใช้งานง่ายขึ้น


ในส่วนของการรองรับผู้ใช้งาน ยังไม่สามารถทำได้ดีพอที่จะรองรับผู้ใช้งานหลายคนทั่วทั้งประเทศพร้อมกันได้
เนื่องจากข้อจำกัดของเซิร์ฟเวอร์ที่ใช้อยู่ในปัจจุบัน ไม่ได้มีประสิทธิภาพมากพอ และทางคณะผู้จัดทำได้วางแผนแนวทางการพัฒนาระบบ
ต่อไปในอนาคตเพื่อให้มีความเสถียรและสามารถขยายขนาด(scalability)ได้โดยไม่ใช้ทรัพยากรมากเกินจำเป็น
\section{\ifenglish Challenges\else ปัญหาที่พบและแนวทางการแก้ไข\fi}

ในการทำโครงงานนี้ พบว่าเกิดปัญหาหลักๆ ดังนี้

1. ระบบเดิมไม่ได้มีการออกแบบเพื่อรองรับการเพิ่มฟีเจอร์ใหม่ตั้งแต่แรก ทำให้การพัฒนามีความล่าช้ากว่าที่ควรเพราะต้องปรับโครงสร้างระบบใหม่
เพื่อให้การพัฒนาระบบในอนาคตเป็นไปได้อย่างราบรื่นขึ้น

2. คณะผู้จัดทำไม่ได้มีพื้นฐานความรู้ด้านการแพทย์ จึงทำให้ไม่ค่อยเข้าใจมุมมองในฐานะทันตแพทย์
ว่าต้องการให้ระบบทำอะไรได้บ้าง 

3. ในช่วงที่ทำการพัฒนาระบบอยู่ มีการใช้งานจากผู้ใช้จริงอยู่ด้วย จึงมีความเสี่ยงที่เมื่อเกิดความผิดพลาด
จะทำให้ข้อมูลจริงเสียหาย

4.ในปัจจุบันพัฒนามาเพื่อรองรับเฉพาะweb browser อาจจะทำให้การแสดงผลไม่ถูกต้องในบางอุปกรณ์ ที่มีขนาดหน้าจอไม่ตรงกับที่ระบบรองรับ

\section{\ifenglish%
Suggestions and further improvements
\else%
ข้อเสนอแนะและแนวทางการพัฒนาต่อ
\fi
}

ข้อเสนอแนะเพื่อพัฒนาโครงงานนี้ต่อไป มีดังนี้

1. ปรับโครงสร้างระบบโดยย้ายระบบไปยังcloud จะทำให้มีscalabilityและreliabilityที่สูงขึ้น

2. มีระบบbackupและrollbackฐานข้อมูล เพื่อป้องกันความผิดพลาดที่อาจจะเกิดขึ้นได้ระหว่างการพัฒนาระบบ

3. ควรมีการทำระบบอีกเวอร์ชันเพิ่มออกมาเพื่อให้ใช้ในอุปกรณ์เคลื่อนที่โดยเฉพาะ เพื่อให้ใช้งานง่าย
และสวยงามขึ้น

4. ควรเพิ่มระบบsecurityให้ดีขึ้นกว่าปัจจุบัน เพราะมีข้อมูลส่วนตัวของผู้ใช้เก็บอยู่ด้วย
อาจเกิดความเสียหายขึ้นได้เมื่อมีผู้ไม่หวังดีมาโจมตีระบบ

5. ควรมีการทำdocumentไว้เพื่อให้ผู้ที่เข้ามาพัฒนาระบบต่อไปสามารถเข้าใจได้ง่ายขึ้น

