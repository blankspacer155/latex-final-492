\maketitle
\makesignature

\ifproject
\begin{abstractTH}

แพลตฟอร์มดิจิทัลสำหรับการตรวจคัดกรองและเฝ้าระวังการเกิดรอยโรคก่อนมะเร็งและมะเร็งช่องปาก 
เป็นเว็บแอปพลิเคชั่นสำหรับรองรับระบบปัญญาประดิษฐ์ (AI) 
เพื่อตรวจคัดกรองและเฝ้าระวังการเกิดรอยโรคก่อนมะเร็งและมะเร็งช่องปาก (Digital Platform for Detecting and Analyzing Oral Potentially Malignant Disorders and Oral Cancer) เบื้องต้นได้ด้วยตนเอง 
กลุ่มผู้ใช้งานของดิจิทัลแพลตฟอร์มนี้จะเป็นทันตแพทย์และประชาชนทั่วไป โดยการยืนยันผลการตรวจของระบบปัญญาประดิษฐ์จะถูกยืนยันผลจากทันตแพทย์ที่เข้าร่วมโครงการ 
เมื่อตรวจสอบพบว่ามีรอยโรคจริงก็ดำเนินการรักษาในขั้นต่อไป

% การเขียนรายงานเป็นส่วนหนึ่งของการทำโครงงานวิศวกรรมคอมพิวเตอร์
% เพื่อทบทวนทฤษฎีที่เกี่ยวข้อง อธิบายขั้นตอนวิธีแก้ปัญหาเชิงวิศวกรรม และวิเคราะห์และสรุปผลการทดลองอุปกรณ์และระบบต่างๆ
% \enskip อย่างไรก็ดี การสร้างรูปเล่มรายงานให้ถูกรูปแบบนั้นเป็นขั้นตอนที่ยุ่งยาก
% แม้ว่าจะมีต้นแบบสำหรับใช้ในโปรแกรม Microsoft Word แล้วก็ตาม
% แต่นักศึกษาส่วนใหญ่ยังคงค้นพบว่าการใช้งานมีความซับซ้อน และเกิดความผิดพลาดในการจัดรูปแบบ กำหนดเลขหัวข้อ และสร้างสารบัญอยู่
% \enskip ภาควิชาวิศวกรรมคอมพิวเตอร์จึงได้จัดทำต้นแบบรูปเล่มรายงานโดยใช้ระบบจัดเตรียมเอกสาร
% \LaTeX{} เพื่อช่วยให้นักศึกษาเขียนรายงานได้อย่างสะดวกและรวดเร็วมากยิ่งขึ้น
\end{abstractTH}

\begin{abstract}

Digital Platform for Detecting and Analyzing Oral Potentially Malignant Disorders and Oral Cancer is web application integrated with AI
 for self detecting oral potentially malilgnant disorders and oral cancer. User group are dentist and general public.
  AI analyzed result will confirmed by dentist participating in the project. 
  If result is verified to be disorders, proceed with treatment in the next step.
% The abstract would be placed here. It usually does not exceed 350 words
% long (not counting the heading), and must not take up more than one (1) page
% (even if fewer than 350 words long).

% Make sure your abstract sits inside the \texttt{abstract} environment.
\end{abstract}

\iffalse
\begin{dedication}
This document is dedicated to all Chiang Mai University students.

Dedication page is optional.
\end{dedication}
\fi % \iffalse

\begin{acknowledgments}

    โครงงานนี้จะไม่สําเร็จลุล่วงลงได้ ถ้าหากไม่ได้รับความกรุณาจาก รศ.ดร.ปฏิเวธ วุฒิสารวัฒนา อาจารย์ที่ปรึกษาโครงงาน ที่ได้สละเวลาส่วนตัวมาให้ความช่วยเหลือแก่โครงงานนี้ โดยได้ให้คําเสนอแนะ แนวคิด ช่องทางการหาความรู้ที่จําเป็นในการทําเว็บแอพลิเคชัน 
    ตลอดจนช่วยตรวจสอบแก้ไขข้อบกพร่องต่างๆ มาโดยตตลอด รวมถึง ผศ.โดม โพธิกานนท์ และ รศ.ดร. ศันสนีย์ เอื้อพันธ์วิริยะกุล ที่ให้คําปรึกษา คําแนะนํา 
    จนทําให้โครงงานนี้มีความสมบูรณ์มากที่สุด
    
    ขอบคุณคณะวิศวกรรมศาสตร์ มหาวิทยาลัยเชียงใหม่ ที่ให้สถานที่ในการทําโครงงาน ทั้งห้องภาควิชา
    วิศวกรรมคอมพิวเตอร์ และสถานที่ต่างๆในภาควิชา และยังให้การสนับสนุนทางด้านงบประมาณ อุปกรณ์
    ต่างๆ ที่จําเป็นต่อการทําโครงงาน
    
    ขอขอบพระคุณผู้ปกครอง เพื่อนและรุ่นพี่ทุกคน ที่ให้คําปรึกษา คําแนะนํา เเละคอยเป็นกําลังใจ
    ให้ตลอดมา ซึ่งเป็นแรงผลักดันให้แก่ผู้จัดทํามีความตั้งใจและมุ่งมั่นในการทํางาน จนโครงงานที่ความสมบูรณ์
    มากที่สุด
    
    นอกจากนี้ผู้จัดทําขอขอบพระคุณอีกหลายๆท่านที่ไม่ได้กล่าวถึง ณ ที่นี้ ที่ได้ให้ความช่วยเหลือตลอดมา
    และสุดท้ายนี้ หากโครงงานนี้มีข้อผิดพลาดประการใด ผู้จัดทําขออภัยมา ณ ที่นี้ และพร้อมน้อมรับด้วยความ
    ยินดี
\texttt{acknowledgment} environment.

\acksign{2020}{5}{25}
\end{acknowledgments}%
\fi % \ifproject

\contentspage

\ifproject
\figurelistpage

\tablelistpage
\fi % \ifproject

% \abbrlist % this page is optional

% \symlist % this page is optional

% \preface % this section is optional
